
\subsection*{Wear Modelling and Introduced Wear Signals}

In the absence of detailed contact mechanics modelling, mechanical wear is represented through physically motivated modifications to an ideal vibration signal. The baseline (non-wear) signal consists of a single periodic component at the fundamental rotation frequency together with additive measurement noise.

Wear is introduced by superimposing additional deterministic components onto this baseline signal. These components are chosen to reflect well-established vibration signatures of mechanical defects in rotating machinery, including nonlinear contact forces, intermittent impacts, and modulation effects. The severity of wear is controlled through a dimensionless wear parameter $\lambda \in [0,1]$, which scales the amplitudes of all wear-related components simultaneously.

\subsubsection*{Mathematical Formulation}

The complete vibration signal incorporating wear effects is expressed as:

\begin{equation}
x(t) = A_0 \sin(2\pi f_0 t) + \sum_{k} A_k(\lambda) \sin(2\pi f_k t + \phi_k) + \eta(t)
\end{equation}

where $A_0 = 1.0$ represents the fundamental amplitude, $f_0 = 25$ Hz is the rotation frequency (1500 RPM), $A_k(\lambda)$ are wear-dependent amplitude scaling factors, and $\eta(t)$ is additive Gaussian noise with standard deviation $\sigma = 0.05 + 0.15\lambda$.

\subsubsection*{Wear Component Specification}

The wear model introduces six distinct frequency components, each representing different physical mechanisms of mechanical degradation:

\begin{table}[h]
\centering
\caption{Wear signal components and their physical interpretation}
\begin{tabular}{|l|c|c|l|}
\hline
\textbf{Component} & \textbf{Frequency} & \textbf{Amplitude} & \textbf{Physical Origin} \\
\hline
2nd Harmonic & $2f_0$ (50 Hz) & $0.3\lambda$ & Nonlinear contact forces \\
3rd Harmonic & $3f_0$ (75 Hz) & $0.2\lambda$ & Higher-order nonlinearity \\
Sub-harmonic & $0.5f_0$ (12.5 Hz) & $0.15\lambda$ & Intermittent contact \\
Upper Sideband & $f_0 + 10$ Hz (35 Hz) & $0.1\lambda$ & Amplitude modulation \\
Lower Sideband & $f_0 - 10$ Hz (15 Hz) & $0.1\lambda$ & Phase modulation \\
\hline
\end{tabular}
\end{table}

\subsubsection*{Physical Justification}

The selection of these specific frequency components is based on established principles of rotating machinery diagnostics:

\begin{itemize}
\item \textbf{Harmonic components} ($2f_0$, $3f_0$) arise from nonlinear contact mechanics when surface irregularities cause time-varying stiffness and damping properties.
\item \textbf{Sub-harmonic components} ($0.5f_0$) result from intermittent contact events, where mechanical clearances cause periodic loss of contact during rotation.
\item \textbf{Sideband frequencies} ($f_0 \pm \Delta f$) emerge from amplitude and phase modulation effects, typically associated with bearing cage dynamics and load variations.
\end{itemize}

\subsubsection*{Wear Progression Characteristics}

As the wear parameter $\lambda$ increases from 0 to 1, the model exhibits several key characteristics observed in real machinery degradation:

\begin{enumerate}
\item \textbf{Spectral enrichment}: The frequency spectrum evolves from a single peak at $f_0$ to a complex multi-peak structure.
\item \textbf{Amplitude growth}: Wear-related components grow proportionally to $\lambda$, while the fundamental remains constant.
\item \textbf{Noise increase}: Broadband noise levels increase with wear, reflecting surface roughening and impact events.
\item \textbf{Nonlinear scaling}: Some diagnostic indicators (e.g., harmonic ratios) exhibit quadratic dependence on $\lambda$, providing enhanced sensitivity.
\end{enumerate}

This simplified yet physically motivated approach enables systematic investigation of signal processing techniques for early fault detection while maintaining computational tractability for academic demonstration purposes.
